% Glossareinträge
\newglossaryentry{glos:quantisierung}{name={Quantisierung}, description={beschreibt den Vorgang, bei dem die Genauigkeit von Modellgewichten reduziert wird, um das Modell kleiner und effizienter zu machen.}}
\newglossaryentry{glos:benchmark}{name={Benchmark}, description={ist ein standardisierter Test oder Vergleichswert, mit dem die Leistung, Qualität oder Effizienz von Systemen, Prozessen oder Produkten objektiv bewertet wird.}}
\newglossaryentry{glos:fine-tuning}{name={Fine-Tuning}, description={ist der Prozess, bei dem ein bereits vortrainiertes Modell mit zusätzlichen, spezifischen Daten nachtrainiert wird, um es besser auf bestimmte Aufgaben oder Anwendungsbereiche abzustimmen.}}
\newglossaryentry{glos:few-shot}{name={Few-Shot}, description={bezeichnet ein Lern- oder Evaluationsszenario, bei dem ein Modell mit nur wenigen Beispielen pro Aufgabe oder Klasse trainiert oder getestet wird.}}
\newglossaryentry{glos:zero-shot}{name={Zero-Shot}, description={bezeichnet ein Szenario, in dem ein Modell Aufgaben lösen soll, ohne zuvor Beispiele dafür gesehen zu haben.}}
\newglossaryentry{glos:inferenz}{name={Inferenz}, description={bezeichnet den Prozess, bei dem ein bereits trainiertes Modell auf Eingaben (Prompts) reagiert und eine Ausgabe (Antwort, Textfortsetzung, etc.) erzeugt.}}
