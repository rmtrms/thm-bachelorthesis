\chapter{Fazit}\label{ch:fazit}

Dieses abschließende Kapitel fasst die zentralen Resultate der Arbeit zusammen und bewertet diese im Kontext der praktischen Anwendbarkeit.
Darauf aufbauend wird ein Ausblick auf weiterführende Forschung und mögliche Optimierungen gegeben.

% ======================================================================================================================

\section{Zusammenfassung der wichtigsten Erkenntnisse und Ergebnisse der Arbeit}

Die vorliegende Arbeit demonstriert, dass auf \glspl{llm} basierende Ansätze eine hochpräzise und effiziente Alternative zu etablierten, regelbasierten Systemen für die Extraktion von Copyright-Informationen darstellen.

Ein zentrales Ergebnis ist die Entwicklung einer formalen Policy, die als normative Grundlage für die Extraktion diente und in zukünftigen Projekten der metaeffekt GmbH als wertvolle Referenz verwendet werden kann.

Zur empirischen Fundierung wurde ein umfangreicher Datensatz mit rund \num{467000} Dateien aus einem Alpine-Linux-Container aggregiert, der für weiterführende Untersuchungen und Modelltrainings zur Verfügung steht.
Damit konnte das Anwendungsszenario 1 zur Generierung eines Testdatensatzes weitgehend erfüllt werden, wobei die zuverlässige Erkennung komplexer Blöcke von Copyright-Statements einen noch fehlenden Validierungsmechanismus erfordert.

Ein systematischer Benchmark, der 30 Open-Source-\glspl{llm} umfasste, identifizierte das Modell \texttt{mistral-small:24b} als die am besten geeignete Basis für den Anwendungsfall.
Die iterative Entwicklung dieses Benchmarks trug maßgeblich zur Implementierung des finalen Prototyps bei.

Durch gezieltes Prompt-Engineering wurde die Extraktionsleistung optimiert, um eine vollständige Abdeckung der Policy zu erreichen, was zu einer Genauigkeit von \num{96,3} \% \textit{Exact Matches} führte.
Eine anschließende Validierung mit ungesehenen Daten bestätigte die hohe Qualität der Lösung für einzelne Copyright-Vermerke, zeigte jedoch zugleich noch Schwächen bei der Verarbeitung von zusammenhängenden Blöcken auf.

Die Experimente zum Fine-Tuning belegen, dass ein spezialisiertes, deutlich kleineres Modell (\texttt{mistral-7b-instruct-v0.3}) bei identischer Genauigkeit eine mehr als dreifache Verarbeitungsgeschwindigkeit im Vergleich zur Prompt-Engineering-Lösung erzielen kann.
Diese Ergebnisse zeigen, dass ein Modell die komplexen Anforderungen der Policy durch implizites Lernen aus Daten verinnerlichen kann, ohne dass detaillierte Anweisungen im Prompt notwendig sind.
Zudem wurde das Potenzial für eine erhebliche Geschwindigkeitssteigerung durch den Einsatz eines noch kleineren Modells (\texttt{qwen2.5-1.5b-instruct}) aufgezeigt, auch wenn dies mit einer spürbaren Reduktion der Genauigkeit verbunden war.
Diese Erkenntnisse schaffen eine wichtige Grundlage für die in den Anwendungsszenarien 2 und 3 geforderte Praxistauglichkeit.

Der entwickelte Prototyp fasst diese Ergebnisse in einer beispielhaften Implementierung zusammen und kann als Ausgangspunkt für die Entwicklung einer Produktivlösung herangezogen werden.

% ======================================================================================================================

\section{Bewertung des entwickelten Prototyps}

Der im Rahmen dieser Arbeit entwickelte Prototyp demonstriert ein erhebliches Potenzial für den praktischen Einsatz im Lizenz-Compliance-Management und erfüllt wesentliche funktionale sowie nicht-funktionale Anforderungen.
Seine Eignung ergibt sich aus einer Kombination von hoher Extraktionsgenauigkeit, technischer Flexibilität und einer klaren Perspektive für den performanten Betrieb.

Ein entscheidender Vorteil für die Praxistauglichkeit ist die ermittelte Extraktionsqualität.
Mit einer Genauigkeit von \num{96,3} \% \textit{Exact Matches} übertrifft der Prototyp die Leistung etablierter regelbasierter Werkzeuge und erfüllt die definierte Anforderung von \num{95} \% (NF1).
Diese hohe Präzision hat das Potenzial, den manuellen Nachbesserungsaufwand erheblich zu reduzieren und die Rechtskonformität in der Software-Lieferkette zu verbessern.
Die Gestaltung des Ausgabeformats (F1), das bewusst an das JSON-Schema des ScanCode-Toolkits angelehnt ist, gewährleistet zudem eine nahtlose Systemintegration (NF11).
Bestehende Prozessketten der metaeffekt GmbH können die generierten Ergebnisse ohne wesentliche Anpassungen weiterverarbeiten, was eine reibungslose Einführung als ergänzende oder ersetzende Komponente ermöglicht.

Die durchgeführten Fine-Tuning-Experimente unterstreichen das Potenzial für einen effizienten, produktiven Einsatz.
Es wurde gezeigt, dass ein spezialisiertes, kleineres Modell bei identischer Genauigkeit eine mehr als dreifache Verarbeitungsgeschwindigkeit erreichen kann.
Dies adressiert Bedenken hinsichtlich der Laufzeitperformance und belegt, dass eine hohe Präzision nicht zwangsläufig mit inakzeptablen Latenzzeiten verbunden ist.

Die Architektur des Systems, die auf einem lokal via Docker betriebenen \gls{llm} basiert, erfüllt zudem die kritische Anforderung des On-Premise-Betriebs.
Dadurch können auch Kundenprojekte bearbeitet werden, bei denen sensible Daten das Unternehmensnetzwerk nicht verlassen dürfen.
Die Verwendung von Modellen unter der Apache-2.0-Lizenz sichert darüber hinaus die notwendige Lizenzkonformität für einen kommerziellen Einsatz (NF10).

Trotz des vielversprechenden Potenzials bestehen für den praktischen Einsatz klare Limitationen und Voraussetzungen.
Die größte Hürde stellen die im Vergleich zu traditionellen Werkzeugen deutlich höheren Hardware-Anforderungen dar.
Die Notwendigkeit leistungsfähiger Systeme mit ausreichend Arbeitsspeicher und GPU-Unterstützung könnte eine Investitionshürde für die Implementierung bei Kunden oder intern darstellen.
Weiterhin zeigten die Validierungen, dass der Prototyp zwar bei einfachen Fällen exzellente Ergebnisse liefert, die Erkennung komplexer Blöcke von Copyright-Statements jedoch noch unzureichend ist.
Vor diesem Hintergrund erscheint ein gestufter Ansatz für die Integration sinnvoll.
Zunächst eignet sich der Scanner für den internen Einsatz bei zeitlich unkritischen Kundenprojekten, um von seiner hohen Genauigkeit zu profitieren, bevor durch weitere Optimierungen ein breiterer Einsatz möglich wird.

% ======================================================================================================================

\section{Ausblick}

Die in dieser Arbeit erzielten Ergebnisse legen eine solide Grundlage für zukünftige Forschung und Entwicklung im Bereich der automatisierten Lizenz-Compliance.
Die identifizierten Potenziale eröffnen vielversprechende Wege, um die hier vorgestellte Lösung zur Produktreife zu führen und die Grenzen bestehender Systeme zu überwinden.

Eine zentrale Weiterentwicklungsperspektive liegt in der Skalierung der Datengrundlage.
Durch die Kombination der leistungsfähigen Prompt-Engineering-Lösung mit einer nachgelagerten, regelbasierten Validierung der Modellausgaben ließe sich ein Datensatz erzeugen, der Hunderttausende hochqualitative Beispiele umfasst.
Ein solcher Datensatz könnte genutzt werden, um ein performantes Instruct-\gls{llm} zu trainieren, das eine hochqualitative Extraktion über alle Aspekte der Policy hinweg ermöglicht.
Durch weiterführende Optimierungen des Trainingsprozesses wäre es denkbar, selbst sehr kleine und ressourcenschonende Modelle zu zufriedenstellenden Ergebnissen zu führen, was die Praxistauglichkeit erheblich steigern würde.

Zukünftige Arbeiten sollten zudem die Ergänzung weiterer Datenquellen in Betracht ziehen.
Auf diese Weise könnten weitreichendere und vielfältigere Beispiele für Copyright-Angaben identifiziert werden, was eine kontinuierliche Schärfung und Erweiterung der Policy ermöglichen würde.
Ein besonderes Potenzial liegt hierbei in der gezielten Integration von mehrsprachigen Quellen.
Dies könnte die Entwicklung eines Systems ermöglichen, das Copyright-Informationen über Sprachgrenzen hinweg zuverlässig extrahiert.
Dies würde eine Fähigkeit darstellen, die derzeit von keinem etablierten Werkzeug angeboten wird und einen erheblichen Mehrwert darstellt.

Darüber hinaus könnte eine breitere Datenbasis die Untersuchung alternativer, gradienten-basierter Machine-Learning-Lösungen ermöglichen, die langfristig eine weniger ressourcenintensive Alternative zur \gls{llm}-Bereitstellung darstellen könnten.

Das Erreichen einer Extraktionsgenauigkeit von über \num{95} \% \textit{Exact Matches} über alle Kategorien hinweg, kombiniert mit einem performanten und effizienten Modell, würde schließlich den Weg für den produktiven Einsatz des Systems ebnen.
Nicht nur intern, sondern auch direkt im Kundenkontext.
Die konsequente Verfolgung dieser Forschungs- und Entwicklungsperspektiven birgt das Potenzial, die automatisierte Lizenz-Compliance nachhaltig zu verbessern und einen neuen Standard an Genauigkeit und Effizienz in diesem wichtigen Feld zu etablieren.