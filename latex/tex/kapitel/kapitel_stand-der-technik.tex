\chapter{Stand der Technik}\label{ch:stand-der-technik}


\section{Das ScanCode-Toolkit}\label{sec:scancode-toolkit}

Das ScanCode-Toolkit ist ein Open-Source Command-Line-Tool, das Softwareprojekte scannt, um automatisch Lizenzen zu erkennen, Copyright-Hinweise zu identifizieren, Autoreninformationen auszulesen und Informationen über Paketabhängigkeiten und Software-Komponenten zu liefern.
Das Toolkit wurde von nexB Inc.\ entwickelt und ist Teil der AboutCode-Initiative.
Laut eigenen Angaben wird es als führende Engine der Branche zur Erkennung von Lizenzen und Urheberrechten anerkannt.
Die Engine des Toolkits wird von den fortschrittlichsten Open-Source und kommerziellen Werkzeugen im Bereich der \gls{sca} eingesetzt.
Im Kontext dieser Arbeit spielt die Funktion des ScanCode-Toolkits, Copyrights zu erkennen, eine zentrale Rolle.
Die Erkennung von Copyrights findet mithilfe einer spezialisierten \gls{nlp} Grammatik statt.
Die anderen Funktionen des ScanCode-Toolkit sind für diese Arbeit nicht weiter relevant und werden deshalb nicht im Detail erläutert\autocite{noauthor_scancode-toolkit-documentation_nodate}.

% TODO: auf Alternativen zu ScanCode eingehen --> z.B. FOSSology

\section{Der ScanCode Service}\label{sec:scancode-service}

Der ScanCode-Service ist eine Entwicklung der Metaeffekt, welche das ScanCode-Toolkit nutzt und erweitert.
Im Gegensatz zum ScanCode-Toolkit nutzt der Service kein \gls{cli}, sondern einen Web-Service der z.B.\ in einem Docker-Container ausgeführt werden kann.
Neben den integrativen Vorteilen eines Services sind auch inhaltliche Verbesserungen Teil seiner Implementierung.

% TODO: hier ein Copyright Beispiel einfügen, welches den Vermerk darstellt.
Die Implementierung des ScanCode-Toolkit bietet interne Schalter, die dazu dienen, Vermerke wie \enquote{all rights reserved} aus dem Copyright auszuschließen.
Der Voreinstellung des ScanCode-Toolkit ist, dass diese Vermerke ausgeschlossen werden, der ScanCode-Service hingegen schließt sie nicht aus.
Außerdem entfernt das ScanCode-Toolkit den Punkt am Ende eines Copyright-Statements, der Service erhält diese Punktsetzung im Extrakt.
Diese Änderungen dienen dazu, die Policy-Konformität des ScanCode-Toolkit zu verbessern\autocite{noauthor_metaeffekt-scancode-service_2025}.

\section{Die Policy}\label{sec:policy}

In Kooperation mit einem Kunden der Metaeffekt wurde eine Policy zur Erkennung und Extraktion von Copyright-Statements definiert.
Die Zusammenarbeit mit dem Kunden ist besonders wertvoll, da die Metaeffekt GmbH mit einem Team von Experten im Bereich des License-Compliance-Managements im engen Austausch steht.
Im Laufe der Arbeit werden Erkenntnisse und Zwischenergebnisse mit diesem Team geteilt und regelmäßig Experteneinschätzungen eingeholt.
Die Policy orientiert sich bei den Kategorien der extrahierten Informationen am ScanCode-Toolkit und unterscheidet nach drei Typen: \enquote{Copyrights}, \enquote{Holders} und \enquote{Authors}:

\begin{itemize}
    \item Copyright Statements (Copyrights) sind Markierungen der Autoren bzw.\ Urheber um die Urheber und den Zeitraum, in dem das Material erstellt wurde, zu identifizieren.
    \item Holders sind Individuen oder Entitäten, die Rechte am Material haben.
    \item Authors sind Individuen oder Entitäten, die das Material verfasst oder Beiträge dazu geleistet haben.
\end{itemize}

Die Policy gliedert sich in eine \gls{cir} und sechs \glspl{cep}.




\section{Analyse bestehender Lösungen und deren Schwächen}\label{sec:analyse-bestehender-losungen}

\section{Vorstellung von Large Language Models und deren Potenzial für die Extraktion unstrukturierter Daten}\label{sec:vorstellung-llm}


\section{Verwandte Arbeiten}\label{sec:verwandte-arbeiten}

