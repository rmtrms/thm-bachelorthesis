\chapter{Anforderungen}\label{ch:anforderungen}

Zur Sicherstellung einer zielgerichteten und nachvollziehbaren Entwicklung werden im Folgenden die Anforderungen an das zu entwickelnde System beschrieben.
Die klare Festlegung dieser Anforderungen ermöglicht eine eindeutige Überprüfung der Zielerreichung und minimiert das Risiko von Missverständnissen im Entwicklungsprozess.
Dabei werden sowohl funktionale- als auch nicht-funktionale Anforderungen strukturiert aufgeführt.
Zur Gewährleistung der Nachverfolgbarkeit wird jede Anforderung nach einem einheitlichen Schema spezifiziert, das eine eindeutige ID, einen Titel, eine Priorität und eine Beschreibung inklusive der Abnahmekriterien umfasst.

\section{Funktionale Anforderungen}\label{sec:funktionale-anforderungen}

\begin{itemize}
    \item Genauigkeit -- das System muss bei der Verarbeitung eines anhand der Policy gepflegten Datensatzes mindestens 80\% Exact-Matches erzielen.
    \item Ergebnisausgabe -- das System muss Ergebnisse in einem strukturierten Format bereitstellen, das für nachgelagerte Verarbeitungsschritte geeignet ist.
    \item Ergebnisvalidierung -- das System muss die Ergebnisse anhand des strukturierten Formats validieren können.
    \item Fehlerbehandlung -- das System muss Fehlerfälle erkennen, protokollieren und aussagekräftige Fehlermeldungen bereitstellen.
    \item Reproduzierbarkeit -- die Ergebnisse des Systems sind reproduzierbar.
    \item Nachvollziehbarkeit -- das System speichert Zwischenergebnisse um Abläufe und eventuelle Fehler nachvollziehbar zu machen.
    \item Vorverarbeitung -- Eingaben werden auf relevante Teile reduziert bevor eine LLM-Anfrage erfolgt.
    \item Batch-Verarbeitung -- das System kann Batches von mindestens 4000 Eingaben sequentiell verarbeiten.
    \item Zeitüberschreitung -- das System muss Zeitüberschreitungen bei LLM-Anfragen erkennen und die Anfrage stoppen können.
\end{itemize}

\section{Nicht-funktionale Anforderungen}\label{sec:nicht-funktionale-anforderungen}

\begin{itemize}
    \item Modellauswahlverfahren -- das eingesetzte Modell muss anhand eines systematischen, nachvollziehbaren und dokumentierten Evaluationsverfahren ausgewählt werden.
    \item Hardware-Kompatibilität -- das System muss auf der vorhandenen Hardware lauffähig und kompatibel sein.
    \item Implementierungssprache -- das System muss in Java implementiert sein, die genaue Version ist nicht vorgegeben.
    \item Lizenzkonformität -- alle verwendeten Tools und Sprachmodelle müssen über eine zum Anwendungsfall passende Lizenz verfügen.
    \item Policy Abdeckung -- die implementierte Lösung muss alle \enquote{in-scope} definierten Policy Aspekte berücksichtigen.
    \item Integration -- das System soll in bestehende Prozesse der metaeffekt integrierbar sein.
\end{itemize}

% Die entwickelte Lösung muss auf der Hardware der metaeffekt lauffähig bzw. kompatibel sein
% Die verwendeten Tools/Sprachmodelle müssen eine zum Anwendungsfall passende Lizenzierung haben
% Die entwickelte Lösung soll alle Policy Teile abdecken, die als in-scope definiert sind
% Die entwickelte Lösung soll ein kleines Kundenprojekt verarbeiten können
% Die entwickelte Lösung muss mindestens 80% exact-matches bei einem anhand der Policy gepflegten Datensatzes erreichen
% Das verwendete Modell muss anhand eines systematischen Verfahrens evaluiert und ausgewählt werden
% Das System soll in den vorhanden Prozess der metaeffekt integrierbar sein
% Das System muss eine Java implementierung sein

\begin{anforderungsliste}
    \nfreq{Implementierungssprache}{Hoch}
        {Der gesamte Quellcode lässt sich fehlerfrei mit einem Java Development Kit der spezifizierten Version (oder höher) kompilieren.}
        {Die Softwarelösung muss vollständig in der Programmiersprache Java implementiert werden, wobei mindestens die Version 8 als Sprach- und Laufzeitumgebung zu verwenden ist.}
\end{anforderungsliste}
