% -------------------------------------------------------
% Abstrakt / Abstract
% Achtung: Wenn Sie im Abstrakt Anführungszeichen verwenden wollen, dann benutzen Sie
%          nicht "` und "', sondern \enquote{}. "` und "' werden nicht richtig
%          erkannt.

% Kurze (maximal halbseitige) Beschreibung, worum es in der Arbeit geht auf Deutsch
\newcommand{\hsmaabstractde}{Die automatisierte Extraktion von Copyright-Informationen ist für die Lizenz-Compliance essenziell, jedoch stoßen etablierte regelbasierte Werkzeuge wie das ScanCode-Toolkit an methodische Grenzen. Ihre starren Muster führen zu unvollständigen und fehlerhaften Ergebnissen, da sie der Vielfalt und Kontextabhängigkeit von Urhebervermerken im Quellcode nicht gerecht werden. Die vorliegende Arbeit konzipiert, implementiert und evaluiert einen neuartigen, auf Large Language Models (LLMs) basierenden Copyright-Scanner, der diese Schwächen adressiert. Hierzu wurde zunächst aus einem Alpine-Linux-Container ein umfangreicher Datensatz mit rund \num{467000} Dateien aggregiert. Darauf aufbauend wurde in einem systematischen Benchmark aus \num{30} Open-Source-LLMs das leistungsfähigste Basismodell identifiziert. Die Untersuchungen zeigen, dass durch gezieltes Prompt-Engineering das Modell \texttt{mistral-small:24b} eine Extraktionsgenauigkeit von \num{96.3} \% exakten Übereinstimmungen (\textit{Exact Matches}) erreichen kann und damit die Leistung regelbasierter Systeme übertrifft. Weiterführende Experimente belegen, dass ein durch QLoRA spezialisiertes, deutlich kleineres Modell (\texttt{mistral-7b-instruct-v0.3}) bei identischer Genauigkeit eine mehr als dreifache Verarbeitungsgeschwindigkeit erzielt. Die Ergebnisse münden in der prototypischen Implementierung eines Scanners, der sich durch ein zum ScanCode-Toolkit kompatibles Ausgabeformat in bestehende Prozessketten integrieren lässt. Die Arbeit demonstriert, dass LLMs eine hochpräzise und effiziente Alternative für die automatisierte Copyright-Extraktion darstellen und das Potenzial haben, die Genauigkeit und Rechtskonformität im Lizenz-Compliance-Management erheblich zu verbessern.}

% Kurze (maximal halbseitige) Beschreibung, worum es in der Arbeit geht auf Englisch
\newcommand{\hsmaabstracten}{Automated extraction of copyright information is essential for license compliance, yet established rule-based tools such as the ScanCode-Toolkit face methodological limitations. Their rigid patterns lead to incomplete and flawed results, as they fail to address the diversity and context-dependency of copyright notices in source code. This thesis designs, implements, and evaluates a novel copyright scanner based on Large Language Models (LLMs) that addresses these weaknesses. At first, a comprehensive dataset of approximately \num{467000} files was first aggregated from an Alpine Linux container. Subsequently, a systematic benchmark of \num{30} open-source LLMs was conducted to identify the most performant base model. The research shows that through targeted prompt engineering, the model \texttt{mistral-small:24b} can achieve an extraction accuracy of \num{96.3} \% in \textit{Exact Matches}, thereby surpassing the performance of rule-based systems. Further experiments demonstrate that a significantly smaller model (\texttt{mistral-7b-instruct-v0.3}), specialized via QLoRA, achieves identical accuracy at more than three times the processing speed. These findings culminate in the prototypical implementation of a scanner that can be integrated into existing process chains through an output format compatible with the ScanCode-Toolkit. The work demonstrates that LLMs represent a highly precise and efficient alternative for automated copyright extraction and have the potential to significantly improve accuracy and legal conformity in license compliance management.}
